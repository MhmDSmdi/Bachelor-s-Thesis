\chapter{جمع‌بندی و کارهای آتی}
\section{جمع‌بندی و نتیجه‌گیری}
با گسترش روزافزون شبکه‌های اجتماعی و استفاده همگانی از آن‌ها، حجم قابل‌توجهی از اطلاعات در میان مردم منتقل می‌شود. این انتقال سریع و گسترده اطلاعات در فضای مجازی، عموم مردم را در معرض دریافت اطلاعات غلط و اخبار جعلی قرار می‌دهد. انتشار اخبار جعلی اثرات منفی جبران‌ناپذیری را برروی افکار عمومی و حتی سیاست‌های کلان دولت‌ها دارد. امروزه با گسترش استفاده از هوش‌مصنوعی و پردازش زبان طبیعی، محققان پژوهش‌های بسیاری را در این حوزه با روش‌هاش گوناگون انجام داده‌اند.

در این پژوهش قدم‌های اولیه برای ایجاد یک سامانه جامع تشخیص اخبار جعلی با استفاده از مفاهیم به‌روز پردازش زبان طبیعی برداشته‌شد. باتوجه به کمبود مجموعه دادگان منسجم اخبار جعلی به زبان فارسی، در گام اول ما یک روش ۵ مرحله‌ای برای استخراج این اخبار ارائه کردیم. سپس با استفاده از پیاده‌سازی این روش به صورت یک ابزار استخراج اخبار جعلی، مجموعه‌ دادگانی به نام «تاج» برای پژوهش‌های آتی و استفاده از آن برای آموزش مدل‌های اولیه ارائه کردیم.

تشخیص اخبار جعلی یک مسئله پیچیده است که حتی انسان به طور معمول با دیدن متن یک خبر به دشواری می‌تواند درمورد جعلی بودن آن اظهار نظر کند. به همین دلیل، برای پیاده‌سازی یک سامانه مقاوم در برابر خطا، باید از فراداده‌های زیادی در کنار متن خبر استفاده کنیم. در این پژوهش ما از فراداده‌های سطح بالای مبتنی بر متن اخبار مانند احساس خبر، موضوعات خبر، دسته خبر و موجودیت‌های نامدار خبر استفاده کردیم و توانستیم بهتر از قبل اخبار جعلی را تشخیص دهیم.

\section{کار‌های آتی}
همانطور که اشاره شد، تشخیص اخبار جعلی نیازمند اطلاعات بسیار زیادی می‌باشد. این اطلاعات به متن اخبار محدود نمی‌شود بلکه شامل ویژگی‌های زمانی، پروفایل کاربری و تارخچه انتشار خبر می‌باشد. یکی از منابع غنی و قابل استفاده برای آموزش یک مدل هوشمند برای تشخیص اخبار جعلی، شبکه‌‌های اجتماعی مانند توئیتر می‌باشد چراکه این شبکه‌ها فراداده‌های بیشتری نسبت به متن خبر در اختیار مدل ما می‌گذارد که باعث می‌شود مدل هوشمند، ماهیت خبر جعلی را به درستی بشناسد و بتواند روی اخبار جدید عملکرد بهتری داشته باشد.